\chapter{Conclusiones y Trabajo Futuro}
\justifying
\section*{Conclusiones}
Como se discutió en el capítulo anterior, los resultados obtenidos
concuerdan con lo anticipado tras revisar la literatura. No obstante,
se esperaban mejores resultados por parte de \textit{Data Banzhaf}.
Este ligero decremento en el rendimiento del método seguramente
esté influenciado por dos factores principales: el primero
es la falta de capacidad de cómputo, que impidió utilizar tamaños
de muestra mayores y obtener un mayor número de muestras para
construir los estimadores. El segundo factor es la ausencia de
la implementación del estimador MSR, que ha demostrado ser
más eficaz en términos de convergencia. Estas limitaciones afectan
a los resultados de todos los métodos en general, pero en especial
al \textit{Data Banzhaf}, siendo una de sus principales ventajas
respecto al resto de métodos basados en semivalores la posibilidad
de utilizar el estimador MSR.

\

Considerando una perspectiva general, se observa que los métodos
de valoración de datos proporcionan buenos resultados en las
tareas evaluadas. Sin embargo, aún no se consideran una
solución realista debido a la alta carga computacional que implican.

\section*{Trabajo Futuro}
Respecto al trabajo futuro, se identifican varias áreas de
mejora potencial para el proyecto:
\begin{itemize}
    \item Implementación del estimador MSR y posterior integración
    en \textit{pyDVL}.
    \item Uso de todos los \textit{datasets}
    citados en \cite{dataBanzhaf}. Aunque ya se han
    desarrollado funciones de lectura y preprocesado
    para estos conjuntos de datos, no se han empleado
    en este trabajo. Dichas funciones están disponibles
    en \href{https://github.com/Fran-AM/TFM}{Fran-AM/TFM}.
    \item Integración de los modelos de \textit{pyTorch}
    mencionados en \cite{dataBanzhaf} en \textit{pyDVL}.
    Para lograr esto, será necesario implementar el
    protocolo SupervisedModel, en todos estos modelos.
    \item Desarrollo de test unitarios para la mejora
    de la calidad del código.
\end{itemize}
