\chapter{Código relevante}
\label{ap:apendice2}
\justifying
Se muestran algunos fragmentos de código que se consideran
interesantes y pueden ayudar a seguir de una forma más
sencilla el desarrollo de este documento.
\newpage



\section{Preprocesado de datos}
\begin{figure}[ht!]
    \begin{lstlisting}
def get_openML_data(
    dataset: str,
    n_data: int,
    flip_ratio: float = 0.0,
)->Dataset:
if dataset in ds_map:
    try:
        data_dict =
        pickle.load(open(openML_path + ds_map[dataset], 'rb'))
        data, target = data_dict['X_num'], data_dict['y']
        target = (target == 1).astype(np.int32)
        data, target = undersamp_equilibration(data, target)
    except FileNotFoundError:
        raise FileNotFoundError(f"File not found for dataset '{dataset}'")
else:
    raise ValueError(f"Dataset '{dataset}' not found")

x_train, x_test, y_train, y_test = train_test_split(
    data, target, train_size=n_data, random_state=42
)

# Normalization
x_mean, x_std= np.mean(x_train, 0), np.std(x_train, 0)
norm = lambda x: (x - x_mean) / np.clip(x_std, 1e-12, None)
x_train, x_test = norm(x_train), norm(x_test)

# Flip labels
if len(y_train.shape) != 1:
    raise ValueError("Expected y_train to be a 1-dimensionalarray, "
                        "but got a different shape.")

n_flip = int(n_data*flip_ratio)
y_train[:n_flip] = 1 - y_train[:n_flip]

return Dataset(
    x_train=x_train,
    y_train=y_train,
    x_test=x_test,
    y_test=y_test
)
    \end{lstlisting}
    \caption{Función para la lectura y preprocesado de los datos de OpenML.}
    \label{fig:preprocesado}
\end{figure}

\newpage

\section{Ejemplo de uso de pyDVL}
\label{sec:ejemplo_pydvl}
\begin{figure}[ht!]
    \begin{lstlisting}
# Contruccion del pyDVL dataset
x_train, x_test, y_train, y_test = train_test_split(
data, target, train_size=n_data, random_state=42
)
pyDataset = Dataset(
    x_train=x_train,
    y_train=y_train,
    x_test=x_test,
    y_test=y_test
    )

# Construccion del modelo y la utilidad
model = MLPClassifier(
hidden_layer_sizes = (md_params["hidden_neurons"],),
learning_rate_init = md_params["learning_rate"],
batch_size = md_params["batch_size"],
max_iter = md_params["max_iter"],

utility = Utility(model, pyDataset, 'accuracy')

# Calculo de los valores de Shapley
values = compute_shapley_values(
    u=utility,
    mode="permutation_montecarlo",
    done=HistoryDeviation(n_steps=100, rtol=0.05),
    truncation=RelativeTruncation(utility, rtol=0.01),
    progress=True,
    n_jobs=6
)
    \end{lstlisting}
\end{figure}





