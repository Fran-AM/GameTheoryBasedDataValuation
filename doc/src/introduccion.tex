\chapter{Introducción}
\justifying

Obtener datos de entrenamiento adecuados suele ser uno de los
desafíos más grandes y costosos en aprendizaje automático (ML).
\index{ML} En muchas aplicaciones prácticas, parte de los datos
puede estar contaminada con ruido debido a valores atípicos o errores
en el etiquetado. Entender qué tipos de datos son más
beneficiosos para el entrenamiento puede guiar a una mejor
adquisición y selección de estos. Debido a estos factores,
la valoración de datos ha emergido como un área esencial de
investigación en el campo del ML.

\

Dentro del contexto del ML, el valor de un dato particular
está intrínsecamente relacionado con otros datos utilizados
en el entrenamiento del modelo. Por ejemplo, el valor de una
muestra específica puede disminuir si se incorporan al conjunto
de entrenamiento otras similares a esta. Por ello,
nos centraremos en un contexto de aprendizaje
supervisado, donde se dispone con un conjunto de datos, un modelo,
y una métrica de error prefijados. Es importante enfatizar que
no estamos definiendo un valor universal para los datos, sino
un valor que depende de los elementos previamente prefijados.

\

Asignar valor a los datos es una tarea
compleja que implica consideraciones multifactoriales.
En este contexto, la teoría de juegos emerge como una
herramienta para abordar esta problemática.
En el ámbito del aprendizaje automático, en lugar de jugadores,
contamos con datos individuales. Si consideramos un problema de
ML como un juego, el objetivo es maximizar
una métrica de rendimiento, como la precisión del algoritmo.
Aquí, el ``beneficio'' es el impacto positivo que un dato
(o conjunto de datos) tiene en el rendimiento del algoritmo.

Una de las piezas más destacadas en este entramado es el
valor de Shapley, procedente de la teoría
cooperativa de juegos. Este concepto permite asignar
a cada jugador (o dato, en nuestro caso) una porción
del valor total del juego, basándose
en sus contribuciones a lo largo de todas las posibles
coaliciones. Es en torno al valor de Shapley que se
desarrollan todos los conceptos que estudiaremos a lo largo de este
trabajo, como es \textit{data Banzhaf}, derivado del
valor de Banzhaf, que ha probado ser el más robusto entre
todos los valores discutidos en la literatura.

\

Sin embargo, aunque métricas como el valor de Shapley o Banzhaf
son teóricamente atractivas, calcularlas en escenarios reales es
prácticamente inviable debido a la explosión combinatoria de
posibles coaliciones. Por esto, estimar los valores
de los datos mediante técnicas  basadas en teoría de juegos
es un gran desafío, ya que implica equilibrar la precisión y
la eficiencia computacional. Los estimadores juegan un papel
fundamental en este aspecto, proporcionando
soluciones prácticas para aproximarse a los valores
exactos sin incurrir en costes prohibitivos.


\subsection*{¿Qué se aporta?}
En este trabajo se han construido las bases para llevar
a cabo una comparativa de calidad de los diferentes métodos
de valoración de datos basados en teoría de juegos y sus
estimadores. Mediante
el desarrollo de una infraestructura de software
escalable, reproducible y bien documentada, que permitirá
continuar con el desarrollo de las pruebas y experimentos.


