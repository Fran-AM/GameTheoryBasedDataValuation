\chapter{Funciones Gamma y Beta}
\label{ap:apendice1}
\justifying
\section{Función Gamma}
Llamaremos función Gamma $(\Gamma)$ a la función definida
en el intervalo $(0,+\infty)$ por
\begin{equation}
    \Gamma(p)=\int_{0}^{+\infty}e^{-t}t^{p-1}\,dt.
\end{equation}

\begin{theorem}
    Para todo $p>0$ se tiene que $\Gamma(p+1)=p\Gamma(p)$.
    
    En particular, $\Gamma(n+1)=n!$ para todo $n\in\mathbb{N}$.
\end{theorem}

\begin{proof}
    Integramos por partes.
    \[
        \Gamma(p+1)=\int_{0}^{+\infty}e^{-t}t^{p}\,dt
        =\left[-e^{-t}t^{p}\right]_{0}^{+\infty}
        +p\cdot\int_{0}^{+\infty}e^{-t}t^{p-1}\,dt
    \]
    Como $\left[-e^{-t}t^{p}\right]_{0}^{+\infty}=0$ al ser
    $p > 0$, tenemos que $\Gamma(p+1) = p\Gamma(p)$.

    \

    Para la segunda afirmación, sabiendo que $\Gamma(1) = \left[
    -e^{-t}\right]_0^{+\infty}=1$, tendremos
    \[
        \Gamma(n+1)=n\Gamma(n)=n(n-1)\Gamma(n-1)=\dots=n!.
    \]   
\end{proof}

\begin{theorem}
    \label{thm:gamma}
    Si $a\in\mathbb{C},$ tal que $Re(a)>0$, y $p>0$, entonces:
    \[
        \int_{0}^{+\infty}e^{-at}t^{p-1}\,dt=\frac{\Gamma(p)}{a^{p}}.
    \]
\end{theorem}

\begin{proof}
    Realizando el cambio de variable $t = au$ en la integral
    de la función $\Gamma$, tenemos:
    \[
        \Gamma(p)=\int_{0}^{+\infty}e^{-t}t^{p-1}\,dt=
        \int_{0}^{+\infty}e^{-au}(au)^{p-1}\cdot a\,du=
        a^{p}\int_{0}^{+\infty}e^{-au}u^{p-1}\,du.
    \]
    Despejando tenemos
    \[
        \frac{\Gamma(p)}{a^{p}}=\int_{0}^{+\infty}e^{-at}t^{p-1}\,dt.
    \]

    Esta igualdad se tiene para todo $a\in\mathbb{R}$, ahora bien, si
    $a\in\mathbb{C}$, con $Re(a)>0$, tenemos que las funciones
    complejas a cada lado de la igualdad son analíticas en $a$
    y coinciden en el eje real positivo, por lo que coinciden en todos
    los puntos en los que sean holomorfas, es decir, al menos en el
    semiplano real positivo.
\end{proof}

\section{Función Beta}
Llamaremos función Beta $Beta$ a la función definida para
todo par de puntos $p,q>0$ por la integral:
\[
Beta(p,q)=\int_{0}^{1}t^{p-1}(1-t)^{q-1}\,dt.    
\]

\begin{theorem}
    La función $Beta$ es simétrica en sus dos variables, es decir,
    $Beta(p,q)=Beta(q,p)$.
\end{theorem}

\begin{proof}
    Realizando el cambio de variable $t = 1-u$ en la integral
    de la función $Beta$, tenemos:
    \[
        Beta(p,q)=\int_{0}^{1}t^{p-1}(1-t)^{q-1}\,dt=
        \int_{0}^{1}(1-u)^{p-1}u^{q-1}\,du.
    \]
    Intercambiando $p$ y $q$ tenemos
    \[
        B(p,q)=\int_{0}^{1}t^{p-1}(1-t)^{q-1}\,dt=
        -\int_{1}^{0}(1-u)^{p-1}u^{q-1}\,du=
        \int_{0}^{1}u^{q-1}(1-u)^{p-1}\,du=B(q,p).
    \]
\end{proof}

\begin{theorem}
    Para todo $p,q>0$ se tiene que
    \[
        Beta(p,q)=\frac{\Gamma(p)\Gamma(q)}{\Gamma(p+q)}.
    \]
\end{theorem}

\begin{proof}
    Realizamos el cambio de variables $v = \frac{t}{1-t}$ en la integral
    que define la función $Beta$, se tiene:
    \[
        Beta(p,q)=\int_{0}^{1}t^{p-1}(1-t)^{q-1}\,dt=
        \int_{0}^{+\infty}\frac{v^{p-1}}{(1+v)^{p+q}}\,dv.
    \]
    Usando la propiedad vista en \ref{thm:gamma}, y tomando
    $\alpha = p + q$ y $a = 1 + v$, tenemos:

    \[
        \frac{1}{(1+v)^{p+q}}=\frac{1}{\Gamma(p+q)} \int_0^{+\infty} x^{p+q-1} e^{-(1+v) x} d x .
    \]
    Sustituyendo esta última igualdad en la expresión anterior
    de la función beta se tiene
    \begin{align*}
        \beta(p, q) & =\int_0^{+\infty} \frac{v^{p-1}}{(1+v)^{p+q}}
        dv=\frac{1}{\Gamma(p+q)} \int_0^{+\infty} \int_0^{+\infty}
        v^{p-1} x^{p+q-1} e^{-(1+v) x}\,dv\,dx= \\
        & =\frac{1}{\Gamma(p+q)} \int_0^{+\infty} x^{p+q-1}
        e^{-x}\left(\int_0^{+\infty} v^{p-1} e^{-v x} dv\right)\,dx.
    \end{align*}
    Volviendo a usar la expresión
    \[
        \int_0^{+\infty} x^{\alpha-1} e^{-a x}\,dx=
        \frac{\Gamma(\alpha)}{a^\alpha}, a, \alpha>0
    \]
    se deduce
    \[
        \beta(p, q)=\frac{1}{\Gamma(p+q)} \int_0^{+\infty}
        x^{p+q-1} e^{-x}\left(\int_0^{+\infty} v^{p-1}
        e^{-v x}\,dv\right) d x=\frac{1}{\Gamma(p+q)}
        \int_0^{+\infty} x^{p+q-1} e^{-x} \frac{\Gamma(p)}{x^p}\,dx
    \]
    y utilizando la expresión de la función gamma, se concluye
    \begin{align*}
        \beta(p, q)&=\frac{1}{\Gamma(p+q)} \int_0^{+\infty} x^{p+q-1}
        e^{-x} \frac{\Gamma(p)}{x^p}\,dx
        &=\frac{\Gamma(p)}{\Gamma(p+q)} \int_0^{+\infty} x^{q-1}
        e^{-x}\,dx=\frac{\Gamma(p) \Gamma(q)} {\Gamma(p+q)}.
    \end{align*}
    
\end{proof}
