\chapter{Funciones Gamma y Beta}
\label{ap:apendice1}
\justifying
\section{Función Gamma}
\begin{definition}
    Llamaremos función Gamma $(\Gamma)$ \index{$(\Gamma)$ }a la
    función definida en el intervalo $(0,+\infty)$ por
    \begin{equation}
        \Gamma(p)=\int_{0}^{+\infty}e^{-t}t^{p-1}\,dt.
    \end{equation}
\end{definition}


\begin{theorem}
    Para todo $p>0$ se tiene que $\Gamma(p+1)=p\Gamma(p)$.
    
    En particular, $\Gamma(n+1)=n!$ para todo $n\in\mathbb{N}$.
\end{theorem}

\begin{proof}
    Integramos por partes.
    \[
        \Gamma(p+1)=\int_{0}^{+\infty}e^{-t}t^{p}\,dt
        =\left[-e^{-t}t^{p}\right]_{0}^{+\infty}
        +p\cdot\int_{0}^{+\infty}e^{-t}t^{p-1}\,dt
    \]
    Como $\left[-e^{-t}t^{p}\right]_{0}^{+\infty}=0$ al ser
    $p > 0$, tenemos que $\Gamma(p+1) = p\Gamma(p)$.

    \

    Para la segunda afirmación, sabiendo que $\Gamma(1) = \left[
    -e^{-t}\right]_0^{+\infty}=1$, tendremos
    \[
        \Gamma(n+1)=n\Gamma(n)=n(n-1)\Gamma(n-1)=\dots=n!.
    \]   
\end{proof}

\begin{theorem}
    \label{thm:gamma}
    Si $a\in\mathbb{C},$ tal que $Re(a)>0$, y $p>0$, entonces:
    \[
        \int_{0}^{+\infty}e^{-at}t^{p-1}\,dt=\frac{\Gamma(p)}{a^{p}}.
    \]
\end{theorem}

\begin{proof}
    Realizando el cambio de variable $t = au$ en la integral
    de la función $\Gamma$, tenemos:
    \[
        \Gamma(p)=\int_{0}^{+\infty}e^{-t}t^{p-1}\,dt=
        \int_{0}^{+\infty}e^{-au}(au)^{p-1}\cdot a\,du=
        a^{p}\int_{0}^{+\infty}e^{-au}u^{p-1}\,du.
    \]
    Despejando tenemos
    \[
        \frac{\Gamma(p)}{a^{p}}=\int_{0}^{+\infty}e^{-au}u^{p-1}\,dt.
    \]

    Esta igualdad se tiene para todo $a\in\mathbb{R}$, ahora bien, si
    $a\in\mathbb{C}$, con $Re(a)>0$, tenemos que las funciones
    complejas a cada lado de la igualdad son analíticas en $a$
    y coinciden en el eje real positivo, por lo que coinciden en todos
    los puntos en los que sean holomorfas, es decir, al menos en el
    semiplano real positivo.
\end{proof}

\section{Función Beta}
\begin{definition}
    Llamaremos función $Beta$ \index{$Beta$} a la función definida
    para todo par de puntos $p,q>0$ por la integral:
    \[
    Beta(p,q)=\int_{0}^{1}t^{p-1}(1-t)^{q-1}\,dt.    
    \]
\end{definition}


\begin{theorem}
    La función $Beta$ es simétrica en sus dos variables, es decir,
    $Beta(p,q)=Beta(q,p)$.
\end{theorem}

\begin{proof}
    Realizando el cambio de variable $t = 1-u$ en la integral
    de la función $Beta$, tenemos:
    \[
        Beta(p,q)=\int_{0}^{1}t^{p-1}(1-t)^{q-1}\,dt=
        \int_{0}^{1}(1-u)^{p-1}u^{q-1}\,du.
    \]
    Intercambiando $p$ y $q$ tenemos:
    \begin{align*}
        Beta(p,q)&=\int_{0}^{1}t^{p-1}(1-t)^{q-1}\,dt=
        -\int_{1}^{0}(1-u)^{p-1}u^{q-1}\,du\\
        &=\int_{0}^{1}u^{q-1}(1-u)^{p-1}\,du=Beta(q,p).
    \end{align*}
        
\end{proof}

\begin{theorem}
    Para todo $p,q>0$ se tiene que
    \[
        Beta(p,q)=\frac{\Gamma(p)\Gamma(q)}{\Gamma(p+q)}.
    \]
\end{theorem}

\begin{proof}
    Realizamos el cambio de variables $t = \frac{v}{1+v}$ en la integral
    que define la función $Beta$, se tiene:
    \[
        Beta(p,q)=\int_{0}^{1}t^{p-1}(1-t)^{q-1}\,dt=
        \int_{0}^{+\infty}\frac{v^{p-1}}{(1+v)^{p+q}}\,dv.
    \]
    Usando el teorema \ref{thm:gamma}, y tomando $a = 1 + v$ y 
    $p + q$ como primer parámetro, tenemos:

    \[
        \frac{1}{(1+v)^{p+q}}=\frac{1}{\Gamma(p+q)} \int_0^{+\infty} e^{-(1+v) t} t^{p+q-1} d t .
    \]
    Sustituyendo esta última igualdad en la expresión anterior
    de $Beta$ se tiene:
    \begin{align*}
        Beta(p, q) &=\int_0^{+\infty} \frac{v^{p-1}}{(1+v)^{p+q}}
        dv\\
        &=\frac{1}{\Gamma(p+q)} \int_0^{+\infty} \int_0^{+\infty}
        v^{p-1} e^{-(1+v) t} t^{p+q-1} \,dv\,dt\\
        & =\frac{1}{\Gamma(p+q)} \int_0^{+\infty} e^{-t} t^{p+q-1}
        \left(\int_0^{+\infty} e^{-t v} v^{p-1} dv\right)\,dt.
    \end{align*}
    Volviendo a usar la expresión
    \[
        \int_0^{+\infty} e^{-a t} t^{p-1} \,dt=
        \frac{\Gamma(p)}{a^p},\ a, p>0
    \]
    se deduce que
    \begin{align*}
        Beta(p, q)&=\frac{1}{\Gamma(p+q)} \int_0^{+\infty}
        e^{-t} t^{p+q-1} \left(\int_0^{+\infty} e^{-t v} v^{p-1}
        \,dv\right) d t\\
        &=\frac{1}{\Gamma(p+q)}\int_0^{+\infty} t^{p+q-1}
        e^{-t} \frac{\Gamma(p)}{t^p}\,dt.
    \end{align*}
        
    Y utilizando la expresión de la función gamma, se concluye
    el resultado buscado
    \begin{align*}
        Beta(p, q)&=\frac{1}{\Gamma(p+q)} \int_0^{+\infty} e^{-t}
        t^{p+q-1} \frac{\Gamma(p)}{t^p}\,dt\\
        &=\frac{\Gamma(p)}{\Gamma(p+q)} \int_0^{+\infty} e^{-t}
        t^{q-1} \,dt=\frac{\Gamma(p) \Gamma(q)} {\Gamma(p+q)}.
    \end{align*}
    
\end{proof}
