\chapter{Marco teórico}
\justifying

% \section*{Introducción}

% aaa

\section{Teoría de juegos}
\subsection*{Definición y conceptos básicos}

La \emph{teoría de juegos} \index{Teoría de juegos} puede ser
entendida como la rama de las matemáticas que analiza situaciones
en las que el resultado para cada jugador o participante depende no
solo de sus propias decisiones, sino también de las tomadas por otros
jugadores. Estas situaciones se conocen como \emph{juegos}. \index{Juego}

\begin{definition}
  Un \emph{juego} es una interacción entre jugadores
  racionales\footnote{Entendemos la racionalidad como el
  hecho de que cada jugador intenta maximizar su propio
  beneficio.}, mutuamente conscientes, en la que las
  decisiones de un jugador impactan en las ganancias de
  otros. Un juego se define por:
  \begin{itemize}
    \item \textbf{Los jugadores que intervienen.} Cada \emph{jugador}
      \index{Jugador} es un agente que tiene a su disposición diversas
      estrategias basadas en las posibles recompensas que
      podría recibir.

    \item \textbf{Las estrategias disponibles para cada jugador.}
      Una \emph{estrategia} \index{Estrategia} es un plan de acción que
      un jugador puede adoptar dentro de un juego. Esta estrategia dicta
      las acciones que tomará en cada situación que se presente
      en el juego.

    \item \textbf{Las ganancias de cada jugador en función de los
      resultados.} Las \emph{ganancias} \index{Ganancia} son
      representaciones de las motivaciones de los jugadores, pudiendo
      representar beneficios, cantidades, o simplemente reflejar la
      conveniencia de los diferentes desenlaces.
  \end{itemize}
\end{definition}

Un tipo particular de juego, que cobra especial importancia
en este trabajo son los juegos cooperativos.

\begin{definition}
  Un \emph{juego cooperativo} \index{Juego!cooperativo} es aquel en
  el que los jugadores pueden comunicarse y negociar con el
  fin de establecer acuerdos vinculantes.
\end{definition}

Los acuerdos mencionados se denominan \emph{coaliciones} \index{Coalición}.
Se trata de grupos de jugadores que eligen actuar juntos para
lograr un objetivo común. Las coaliciones pueden
variar desde la compuesta por todos los jugadores hasta la que
incluye a un único jugador.

\

Un juego cooperativo se define completamente a través de su
conjunto de jugadores $N$ \index{$N$} y de su función característica.
Esta función establece la relación entre la formación de
coaliciones y los beneficios que obtienen los jugadores,
asignando a cada posible coalición (es decir, a cada
subconjunto del conjunto potencia de $N$, denotado $2^N$) \index{$2^N$}
un beneficio específico que pueden alcanzar.

\begin{definition}
  Una \emph{función característica} $v$ \index{$v$} es una función
  \begin{align*}
    v:2^N& \longrightarrow \mathbb{R}\\
    S&\longmapsto v(S)
  \end{align*}

  que asigna a cada posible coalición $S$ \index{$S$} la máxima
  ganancia que sus jugadores pueden obtener,
  independientemente de lo que haga el resto de
  jugadores.
\end{definition}

Un interrogante esencial emerge tras establecer coaliciones
y ganancias: ¿cómo se reparten los beneficios entre los
miembros de la coalición? Esta pregunta nos
conduce directamente al valor de Shapley.

\subsection{El valor de Shapley}
\label{sec:valorShapley}

A partir de ahora, consideraremos la situación de un
juego cooperativo. Supondremos un conjunto de $N$ jugadores e
identificaremos dicho juego mediante su función
característica $v$.

\

El \emph{valor de Shapley} \index{Valor de Shapley} es un concepto
de teoría de juegos que asigna de manera equitativa las ganancias
entre los miembros de una coalición. Propuesto
por Lloyd Shapley en 1952 \cite{shapleyValue},
se fundamenta en los siguientes axiomas:

\begin{itemize}
  \item \textbf{Simetría}. Si dos jugadores son simétricos, es decir,
  si su contribución a cualquier coalición es la misma,
  entonces poseen el mismo valor. \index{Simetría}
  $$
  \text{Si } \forall S \subseteq N \setminus \{i, j\},\ 
  v(S \cup \{i\}) = v(S \cup \{j\}) \implies \phi_{Shapley}(i;v) =
  \phi_{Shapley}(j;v).
  $$

  \item \textbf{Eficiencia}. El valor total producido por la coalición
  conformada por todos los jugadores se distribuye entre los
  jugadores. Es decir, \index{Eficiencia}
  $$
  v(N)=\sum_{i\in N}\phi_{Shapley}(i;v).
  $$

  \item \textbf{Linealidad}. Dados dos juegos $u$ y $v$, el valor del
  juego $u+v$ es la suma de los valores de cada juego. Es decir,
  \index{Linealidad}
  $$
  \phi_{Shapley}(i;u+v)=\phi_{Shapley}(i;u)+\phi_{Shapley}(i;v)\ \forall i \in N.
  $$

  \item \textbf{Jugador nulo}. Los jugadores que no aportan a ninguna
  coalición tendrán valor nulo. Es decir, \index{Jugador!nulo}
  $$
  \text{Si } \forall S \subseteq N \setminus \{i\},\ 
  v(S \cup \{i\}) = v(S) \implies \phi_{Shapley}(i;v) = 0.
  $$
\end{itemize}

El siguiente teorema, extraído de \cite{shapleyValue}, prueba
que se trata del único método de valoración de datos que satisface
estos axiomas.

\begin{theorem}
  Existe una única función $\phi_{Shapley}$ \index{$\phi$!$\phi_{Shapley}$}
  satisfaciendo los axiomas de simetría, eficiencia, linealidad y
  jugador nulo, y viene dada por la fórmula:
  \begin{align*}
    \phi_{Shapley}(i;v)&=\sum_{S\subseteq N}
    \frac{(s-1)!(n-s)!}{n!}(v(S)-v(S \setminus \{i\}))\\
    &= \sum_{S\subseteq N \setminus \{i\}} \frac{s!(n-s-1)!}
    {n!}(v(S\cup \{i\})-v(S)).
  \end{align*}
  Donde $n = |N|$ y $s = |S|$ son los cardinales de $N$ y $S$
  respectivamente. \index{$n$} \index{$s$}
\end{theorem}

La demostración detallada se encuentra en la sección 3
de \cite{shapleyValue}.

\

Más adelante nos será útil expresar el valor de Shapley de
una forma alternativa, que nos permitirá relacionarlo mejor
con otros conceptos.

\begin{proposition}
  El valor de Shapley puede ser expresado como:
  \begin{equation}
    \label{shapleyFormula}
    \phi_{Shapley}(i;v)=\frac{1}{n}\sum_{k=1}^n \binom{n-1}{k-1}^{-1} 
    \sum_{\substack{S \subseteq N \setminus \{i\} \\ |S|=k-1}}(v(S\cup\{i\})-v(S)).
  \end{equation}
\end{proposition}

\begin{proof}
  Teniendo en cuenta que
  \[
  \binom{n-1}{k-1}^{-1} = \frac{(k-1)!(n-k)!}{(n-1)!},
  \]
  podemos reescribir la ecuación \ref{shapleyFormula} como:
  \begin{align*}
    \phi_{Shapley}(i,v) &= \frac{1}{n} \sum_{k=1}^{n} \frac{(k-1)!(n-k)!}{(n-1)!}
    \sum_{\substack{S \subseteq N \setminus \{i\} \\ |S|=k-1}}(v(S\cup\{i\})-v(S))\\
    &= \sum_{k=1}^{n} \frac{(k-1)!(n-k)!}{n!}
    \sum_{\substack{S \subseteq N \setminus \{i\} \\ |S|=k-1}}(v(S\cup\{i\})-v(S))\\
    & = \sum_{S\subseteq N \setminus \{i\}} \frac{s!(n-s-1)!}
    {n!}(v(S\cup \{i\})-v(S)).
  \end{align*}

\end{proof}

% Como observación final, es importante señalar que se toman en cuenta
% todas las coaliciones que no incluyen al jugador en cuestión.
% Para cada una de esas coaliciones, se determina la
% \emph{contribución marginal} del jugador, la cual
% representa la diferencia entre la ganancia de la coalición
% con y sin el jugador.
% El Valor de Shapley, en última instancia, se calcula como
% el promedio de todas estas contribuciones marginales.


\subsection{Semivalores}

Los semivalores son una generalización del valor de Shapley,
que surgen al relajar el axioma de eficiencia. Así, permiten
una variedad más amplia de métodos de valoración en el estudio
de juegos cooperativos, siendo especialmente útiles en escenarios
donde no es necesario o deseado que la totalidad de la recompensa
sea distribuida. \index{Semivalor}

La clave de los semivalores es que asignan pesos a las
coaliciones basándose en el tamaño de estas. Estos pesos
son utilizados para calcular la contribución marginal promedio
de cada jugador. A diferencia del valor de Shapley, que es
único dada su definición basada en axiomas \cite{shapleyValue},
hay múltiples semivalores posibles dependiendo de cómo se
determinen los pesos. Esta variabilidad en los semivalores queda
formalizada en el siguiente teorema:

\begin{theorem}{Representación de semivalores \cite{Dubey2}.}
  
  Una función $\phi$ \index{$\phi$} es un semivalor sii
  existe una función peso $w:\{1,\dots,n\} \rightarrow
  \mathbb{R}$ tal que \index{Función peso} \index{$w$}
  $$\sum_{k=1}^n \binom{n-1}{k-1}w(k) = n,$$
  que permite representar $\phi$ mediante la expresión:
  $$
  \phi(i;v) = \sum_{k=1}^n \frac{w(k)}{n}
  \sum_{\substack{S \subseteq N \setminus \{i\} \\ |S|=k-1}}
  (v(S\cup \{i\} - v(S))).
  $$
\end{theorem}

La demostración se puede consultar en \cite{Dubey2}.

\

Este teorema establece una biyección entre semivalores y
funciones de peso. Esta correspondencia uno a uno
nos permite identificar y estudiar semivalores a través
de sus funciones de peso. Además de esto, proporciona
una fórmula general para representar cualquier semivalor
en términos de su función de peso.

Finalmente, es importante destacar que el valor de Shapley
es un caso particular de semivalor. En este caso la función
de peso será $w_{Shapley} = \binom{n-1}{k-1}^{-1}$.
\index{$w$!$w_{Shapley}$}


\subsection{El valor de Banzhaf}

El \emph{valor de Banzhaf}, también denominado índice de poder
de Banzhaf \cite{banzhaf}, es una métrica de teoría de juegos
cooperativos que busca cuantificar el poder e influencia de un
jugador dentro de una coalición. \index{Valor de Banzhaf}
Este índice fue introducido por John F. Banzhaf III en 1965,
con la finalidad de ofrecer una herramienta analítica que
pudiese determinar el poder de influencia de un jugador,
especialmente en escenarios de votación ponderada.

Para comprender mejor la esencia y la utilidad del valor de
Banzhaf, es fundamental familiarizarse con ciertos conceptos
relacionados:

\begin{itemize}
  \item \emph{Sistema de votación ponderado}: es un sistema de
  votación en el que cada jugador tiene un peso o poder de voto
  particular. Para que una propuesta sea aprobada, la suma de los
  pesos de los jugadores que votan a favor debe superar un
  umbral o cuota establecida.

  \item \emph{Jugador pivote}: ee considera que un jugador actúa
  como pivote si, al modificar su voto de negativo a positivo,
  la propuesta es aprobada. Sin embargo, si se abstuviera o
  mantuviera su voto en contra, la propuesta sería rechazada.

  \item \emph{Índice de poder de Banzhaf}: este índice mide la
  frecuencia con la que un jugador se convierte en pivote.
  Es importante destacar que el poder de un jugador no siempre
  es directamente proporcional a su peso en la votación.
\end{itemize}

\begin{definition}
  El \emph{valor de Banzhaf} de un jugador $i$ en un juego
  cooperativo $v$ viene dado por la expresión:
  \begin{equation}
    \label{ec:banzhafFormula}
    \phi_{Banzhaf}(i;v) = \frac{1}{2^{n-1}} \sum_{S
    \subseteq N \setminus \{i\}} [v(S \cup \{i\}) - v(S)].
  \end{equation}
\end{definition}

A pesar de sus similitudes con el valor de Shapley, el índice
de Banzhaf se distingue en su enfoque y la forma de asignar
poder a los jugadores.
Mientras que el valor de Shapley se basa en contribuciones
marginales promediadas, el índice de Banzhaf se enfoca en la
capacidad de un jugador de influir en el resultado final de
una votación. Específicamente, es un semivalor
con un peso asociado dado por $w_{Banzhaf} = \frac{1}{2^{n-1}}$.
\index{$w$!$w_{Banzhaf}$}

\newpage
\section{Valoración de datos}

Hoy en día, el dato representa uno de los recursos más
valiosos en el mundo para negocios, gobiernos y particulares.
La toma de decisiones basada en datos está presente en
casi todos los ámbitos de la sociedad, desde la medicina
predictiva hasta la publicidad personalizada. Debido a
esto, la habilidad para determinar el valor de un dato
se ha vuelto indispensable. Es aquí donde entra
en juego la valoración de datos.

\

Cuando nos referimos al valor de un dato, es esencial
comprender que dicho valor no es unidimensional. Un dato
puede ser valorado desde diferentes perspectivas y
categorizado basándose en diversas cualidades:

\begin{enumerate}
  \item \textbf{Valor Intrínsico vs. Extrínseco}:
  \begin{itemize}
    \item \emph{Valor intrínseco}: se refiere al valor
    inherente al propio dato, basado en su precisión y calidad.
    Este valor es independiente del uso que se le dé al dato.

    \item \emph{Valor extrínseco}: se corresponde al valor
    que se le atribuye al dato en función de su uso.
    Depende, pues, del contexto en el que se use el dato.
  \end{itemize}

  \item \textbf{Valor Directo vs. Indirecto}:
  \begin{itemize}
    \item \emph{Valor directo}: alude al beneficio
    inmediato que se obtiene de un dato, como podría ser
    al venderlo.

    \item \emph{Valor indirecto}: se refiere al
    beneficio derivado del uso estratégico del dato.
  \end{itemize}
\end{enumerate}

En este trabajo nos centraremos en estudiar el valor extrínseco
e indirecto de los datos. Esta investigación nos permitirá,
posteriormente, determinar el valor directo de los datos y
detectar posibles problemas en su valor intrínseco.

\

A pesar de que la valoración de datos es un concepto
multifacético que depende de varios factores, nos
ajustaremos al enfoque propuesto en \cite{dataShapley,betaShapley},
el cual se compone de tres elementos esenciales:

\begin{enumerate}
  \item  Denominaremos $N$ al \emph{conjunto prefijado de datos
  de entrenamiento}, siendo $N = \{ (x_i, y_i) \}_1^n$.
  Aquí, $x_i$ hace referencia a las características del dato
  $i$-ésimo, e $y_i$ a su categoría en problemas de
  clasificación o su valor en problemas de regresión.
  
  \item El \emph{algoritmo de aprendizaje}  $\mathcal{A}$,
  será tratado como una caja negra que toma un conjunto de
  entrenamiento $N$ y genera un predictor $f$.
  \index{$\mathcal{A}$} \index{$f$} 
  
  \item La \emph{función de utilidad} $v$ es una aplicación
  que asigna a cada subconjunto de $N$ un valor, reflejando
  la utilidad de ese subconjunto. Para problemas de clasificación,
  la opción común para $v$ es la precisión del modelo
  entrenado con el subconjunto dado, es decir
  $v(S) = acc (\mathcal{A} (S))$. Sin pérdida de generalidad
  asumiremos a lo largo del documento que $v (S) \in [0, 1]$
  para cualquier $S \subseteq N$. \index{Función de utilidad}
\end{enumerate}

Por lo tanto, podemos entender la valoración de datos como el
proceso de asignar un valor a cada dato del conjunto $N$,
reflejando su contribución en el entrenamiento del modelo.
Cada uno de estos valores estará determinado por $N,
\mathcal{A}$ y $v$, pero por simplicidad lo expresaremos
como $\phi(i ; v)$. A estas puntuaciones se les denomina
\textit{data values}. \index{Data values}

Los distintos enfoques que seguiremos para calcular estos
\textit{data values} se desarrollan a continuación.

\subsection*{LOO Error}

El método más sencillo para valorar datos consiste en medir la
contribución de un punto individual al desempeño global del
conjunto de entrenamiento: \index{$\phi$!$\phi_{loo}$}
\[
  \phi_{{loo}} (i ; v) = v (N) - v (N \setminus \{i\}).
\]
Este método es conocido como \textit{leave-one-out} (LOO). \index{LOO}
Para calcular el valor exacto de los valores LOO para
un conjunto de entrenamiento de tamaño $N$, sería
necesario reentrenar el modelo $n$ veces. Este procedimiento
resulta poco práctico cuando el tamaño del conjunto de datos
es considerablemente grande \cite{looFuck}.

\subsection*{Data Shapley}
En la sección \ref{sec:valorShapley} se presentó
el valor de Shapley, y en \cite{shapleyValue}, se propone
el desarrollo de un método para lograr una valoración equitativa:
\textit{Data Shapley} \index{Data Shapley}. El nombre se debe al
hecho de que las condiciones de la valoración equitativa coinciden
con los axiomas del valor de Shapley. La fórmula de \textit{Data Shapley} es
la misma que la vista en la ecuación \ref{shapleyFormula}

% La siguiente proposición es la que da nombre a ...

% \begin{proposition}
%   Cualquier $\phi (N, \mathcal{A}, V)$ que satisfaga las condiciones
%   anteriores será de la forma
%   \[
%   \phi_i = C \sum_{S \subseteq N \setminus \{ i \}}
%   \frac{V (S \cup \{ i\}) - V (S)}{\binom{n - 1}{| S |}} .
%   \]
%   Dónde el sumatorio contempla todos los subconjuntos de $N$ que no
%   contienen a $i$ y $C$ es una constante arbitraria. Llamaremos a $\phi_i$ el
%   Valor de Shapley asociado al dato $i$.
% \end{proposition}

% \begin{proof}
%   Se puede consultar en {\cite{shapleyValue}}.
% \end{proof}


\subsection*{Beta Shapley}
En \cite{betaShapley} proponen  utilizar la función Beta con
parámetros positivos $(\alpha,\beta)$ para definir
un caso particular de semivalor, con la siguiente función de peso:
\begin{equation*}
  \begin{split}
    w_{\alpha, \beta}(j) :&= n \int_{0}^{1} t^{j-1}(1-t)^{n-j}
    \frac{t^{\beta-1}(1-t)^{\alpha-1}}{\text{Beta}(\alpha,\beta)}dt\\
    &= n\frac{\text{Beta}(j+\beta-1, n-j+\alpha)}
  {\text{Beta}(\alpha,\beta)}.
  \end{split}
\end{equation*}
 
Donde $\text{Beta}(\alpha,\beta) = \frac{\Gamma(\alpha)\Gamma(\beta)}
{\Gamma(\alpha+\beta)}$ es la función Beta, y $\Gamma(\circ)$ es
la función Gamma. Utilizando algunas de las propiedades vistas
en el apéndice \ref{ap:apendice1} podemos simplificar la expresión
anterior a: \index{Beta} \index{$\Gamma$} \index{$w$!$w_{\alpha, \beta}$}
\begin{equation*}
  \begin{split}
  \label{eq:pesoBeta}
  w_{\alpha, \beta}(j) = n\frac{\prod_{k=1}^{j-1}(\beta+k-1)
  \prod_{k=1}^{n-j}(\alpha+k-1)}
  {\prod_{k=1}^{n-1}(\alpha+\beta+k-1)}.
  \end{split}
\end{equation*}

Con esto, el nuevo semivalor propuesto, al que llamaremos
Beta($\alpha, \beta$)-Shapley, queda definido como:
\index{$\phi$!$\phi_{Beta}$}

\begin{equation*}
  \label{eq:betaShapley}
  \phi_{Beta}(i;v) := \frac{1}{n} \sum_{j=1}^{n} 
  w_{\alpha, \beta}(j) \sum_{\substack{S \subseteq N \setminus \{i\}
  \\ |S|=j-1}}
  (v(S \cup \{i\})-v(S)).
\end{equation*}

Los hiperparámetros ($\alpha, \beta$) determinan la
distribución de pesos para cada elemento de $N$.
Por ejemplo, cuando $\alpha = \beta = 1$ tendremos,
$w_{1,1}(j) = \binom{n-1}{j-1}^{-1}$ para todo $j \in N$,
dando el peso uniforme en las contribuciones marginales,
es decir, $\text{Beta}(1,1)$-Shapley es exactamente
el valor de Shapley de los datos originales.
Cuando $\alpha \geq \beta = 1$, el peso normalizado asigna
grandes pesos a conjuntos de pequeña cardinalidad y elimina
el ruido de los conjuntos de cardinalidad mayor.
Por el contrario, $\text{Beta}(1,\beta)$ pone más pesos en
la gran cardinalidad y se acerca a LOO a medida que $\beta$
aumenta.

\subsection*{Data Banzhaf}
Se denomina \textit{Data Banzhaf} ($\phi_{Banzhaf}$)
\index{$\phi$!$\phi_{Banzhaf}$} a la aplicación del valor de
Banzhaf (Ecuación \ref{ec:banzhafFormula}) en el
contexto de valoración de datos. Este semivalor presenta una
serie de ventajas que lo hacen especialmente interesante. Dichas
propiedades serán estudiadas en detalle en secciones
siguientes. \index{Data Banzhaf}

\newpage
\section{Midiendo la robustez}

En diversas situaciones, como la selección de datos, el orden
de los \textit{data values} es lo que aporta valor \cite{betaShapley}.
Un ejemplo podría ser el filtrado de datos de baja calidad.
El escenario ideal sería aquel en el que, incluso
estando perturbada la función de utilidad, se preserva
el mismo orden de \textit{data values}.

En este contexto, la robustez alude a la resistencia
de los métodos de valoración de datos ante perturbaciones
o ruido. Del mismo modo que un modelo de aprendizaje robusto
debería resistir entradas ruidosas, un método de valoración de
datos robusto debería conservar el orden de los \textit{data values} a pesar
del ruido intrínseco de los algoritmos
de aprendizaje automático.

\

Ahora, vamos a establecer los conceptos requeridos para
formalizar y medir la robustez. Recordemos que un semivalor
está determinado por su función de peso $w$. Así, definimos
la diferencia escalada como:


\begin{definition}
  Dados $i,j \in N$, la diferencia escalada \index{Diferencia escalada}
  entre los \textit{data values} $\phi(i;v)$ y $\phi(j;v)$
  se define como: \index{$D_{i,j}$}
  \begin{align*}
    D_{i,j}(v,w)&:= n(\phi(i;v)-\phi(j;v))\\
    &=\sum_{k=1}^{n-1} (w(k)+w(k+1)) \binom{n-2}{k-1}
    \Delta_{i,j}^k(v).
  \end{align*}
  Donde $\Delta_{i,j}^k(v):=\binom{n-2}{k-1}^{-1} \sum_{\substack{
    S \subseteq N \setminus \{i,j\} \\ |S|=k-1}}
    (v(S \cup \{i\})-v(S \cup \{j\}))$ \index{$\Delta_{i,j}^k$}
  se denomina \emph{distinguibilidad promedio entre $i$ y $j$} en
  subconjuntos de tamaño $k$ usando una función de utilidad
  sin ruido $v$.
\end{definition}

Sea ahora $\hat{v}$ un estimador de $v$. Es fácil ver que
\index{$\hat{v}$} $\hat{v}$ y $v$ generarán diferentes
\textit{data values} para un par de puntos $i$ y $j$ sii
$D_{i,j}(v,w)D_{i,j}(\hat{v},w) \leq 0$.

Se podría pensar
inicialmente en definir la robustez de un semivalor como
la menor cantidad de ruido $||\hat{v}-v||$ que alteraría
el orden de los \textit{data values}. Sin embargo, una
definición así dependería de la función de utilidad original $v$.
Si la función original $v$ no es capaz de diferenciar
dos puntos $i$ y $j$ ($\Delta_{i,j}^{k}(v)\simeq 0\ \forall
k=1,\dots,n-1$), entonces $D_{i,j}(v,w)$ será
casi 0, y cualquier mínima perturbación podría 
modificar el orden entre $\phi(i;v)$ y $\phi(j;v)$. 

Por ello, para definir de formar razonable la
robustez de un semivalor, debemos considerar solo las
funciones de utilidad que sean capaces de distinguir
entre $i$ y $j$.

\begin{definition}
  Diremos que un par de puntos $(i,j)$ son $\tau$-distinguibles por
  la función de utilidad $v$ si, y solo si, $\Delta_{i,j}^{k}(v) \geq \tau$
  para todo $k \in \{1,\dots,n-1\}$.
\end{definition}

Sea ahora $\mathcal{V}_{i,j}^{k}$ el conjunto de todas las funciones
de utilidad $v$ que son capaces de $\tau$-distinguir $i$ y $j$. 
Usando la definción anterior, podemos caracterizar la robustez de
un semivalor mediante su \textit{safety margin}, que representa la
menor cantidad de ruido $||\hat{v} - v||$ que, al añadirse,
invertiría el orden de los \textit{data values} de al menos un
par de puntos $(i,j)$, para al menos una función de utilidad $v \in
\mathcal{V}_{i,j}^{k}$.

\begin{definition}
  Dado $\tau > 0$, definimos el \emph{safety margin} \index{Safety margin}
  de un semivalor para un par de puntos $i,j \in N$ como:
  \index{$\text{Safe}_{i,j}$}
  \begin{equation*}
    \text{Safe}_{i,j}(\tau;w):=\min_{v \in \mathcal{V}_{i,j}^{(\tau)}}
    \min_{\hat{v} \in V}
    ||\hat{v} - v||.
  \end{equation*}
  Donde $V = \{\hat{v}:D_{i,j}(v;w)D_{i,j}(\hat{v};w)\leq 0\}$.

  El \emph{safety margin} de un semivalor es: \index{$\text{Safe}$}
  \begin{equation*}
    \text{Safe}(\tau;w):=\min_{i,j \in N, i \neq j} \text{Safe}_{i,j}(\tau;w).
  \end{equation*}
\end{definition}

% La intuición detrás del \textit{safety margin} es que muestra
% la máxima cantidad de ruido que puede ser añadida a un semivalor sin que
% se altere el orden de los \textit{data values} de ningún par de
% puntos que fuera distinguible por la función de utilidad original.


% Corregido hasta aquí por GPT.
\subsection{Robustez del valor de Banzhaf}
Los resultados aquí mostrados pertenecen a la sección 4
de \cite{dataBanzhaf}.

\begin{theorem}
  Para cualquier $\tau > 0$, el valor de Banzhaf
  alcanza el mayor \textit{safety margin}
  \[
  \text{Safe}(\tau;w_{Banzhaf})=\frac{\tau}{2^{\frac{n}{2}-1}}.  
  \]
  de entre todos los semivalores.
\end{theorem}

La demostración se puede consultar en el apéndice
C de \cite{dataBanzhaf}.

\

Intuitivamente, este resultado es consecuencia de cómo los
semivalores asignan diferentes pesos en función del
tamaño de los subconjuntos evaluados. Así, es posible
construir una perturbación de la función de utilidad
que maximice la influencia sobre el semivalor
correspondiente, introduciendo ruido en los subconjuntos
con mayor peso asignado. De ahí que la estrategia
óptima para robustecer sea asignar pesos uniformes a
todos los subconjuntos, tal como lo hace el valor de Banzhaf.

Además, se puede demostrar que el valor de Banzhaf es
el semivalor más robusto en el sentido de que el ruido
de la utilidad afecta mínimamente a los cambios en los
\textit{data values}. En concreto, el valor de
Banzhaf alcanza la menor constante de Lipschitz $L$
tal que $||\phi(v)-\phi(\hat{v})|| \leq L||v-\hat{v}||$,
para todos los posibles pares de funciones de utilidad
$v$ y $\hat{v}$.

\begin{theorem}
  El valor de Banzhaf, con $w_{Banzhaf}(k) = \frac{n}{2^{n-1}}$,
  logra la menor constante de Lipschitz,
  $L = \frac{1}{2^{\frac{n}{2}-1}}$ de entre todos los semivalores.
\end{theorem}

La demostración se puede consultar en el apéndice
C.4 de \cite{dataBanzhaf}.

\

\subsection{Estimación eficiente}

Dado que es prácticamente imposible calcular los
\textit{data values} exactos en métodos de valoración
de datos basados en semivalores, debido a la necesidad
de un número exponencial de evaluaciones de la función
de utilidad, se debe recurrir a métodos de aproximación.
A continuación, introducimos el concepto de error
asociado a un estimador, el cual nos será de utilidad
a la hora de comparar distintos estimadores.

\begin{definition}
  Un estimador de un semivalor $\hat{\phi}$ es \index{$\hat{\phi}$}
  una $(\epsilon,\delta)$-aproximación del semivalor $\phi$ 
  en norma $l_p$ si, y solo si,
  \index{$(\epsilon,\delta)$-aproximación}
  \[
  P_{\hat{\phi}}[||\hat{\phi}-\phi||_p\leq \epsilon] \geq 1-\delta.
  \]
  Donde la aleatoriedad se da en la construcción del estimador.
\end{definition}

\subsubsection*{Estimador Simple de Montecarlo}
El valor de Banzhaf (Ecuación \ref{ec:banzhafFormula})
puede ser reformulado como:
\begin{equation*}
  \label{simpleMontecarlo}
  \phi_{Banzhaf}(i;v) = \mathbb{E}_{S \sim \text{Unif}
  (2^{N\setminus \{i\}})} [v(S \cup \{i\})-v(S)].
\end{equation*}

Donde $\mathbb{E}$ \index{$\mathbb{E}$} denota la esperanza.

Apartir de esto, un método de Montecarlo directo para estimar
$\phi_{Banzhaf}(i;v)$ consistiría
en generar muestras uniformes de $\mathcal{S}_i
\subset 2^{N \setminus \{i\}}$ y calcular:
\begin{equation}
  \label{ec:simpleMontecarloEstimator}
  \hat{\phi}_{MC}(i;v) = \frac{1}{|\mathcal{S}_i|}\sum_{S \in
  \mathcal{S}_i} [v(S_j \cup \{i\})-v(S_j)].
\end{equation}

Al repetir este proceso para cada $i \in N$, obtendremos
el estimador $\hat{\phi}_{MC} = [\hat{\phi}_{MC}(1),\dots,
\hat{\phi}_{MC}(n)]$. \index{$\hat{\phi}_{MC}$}


\begin{theorem}
  El estimador de Montecarlo simple $\hat{\phi}_{MC}$
  es una $(\epsilon,\delta)$-aproximación de $\phi_{Banzhaf}$
  en norma $l_p$ con $\mathcal{O}(\frac{n^2}{\epsilon^2}
  \log(\frac{n}{\delta}))$ evaluaciones de $v$, y
  $\mathcal{O}(\frac{n}{\epsilon^2}
  \log{\frac{n}{\delta}})$ evaluaciones de $v$ en la norma
  $l_{\infty}$.
\end{theorem}

La demostración se puede consultar en el apéndice
C.1.2 de \cite{dataBanzhaf}.

\

\subsubsection*{Estimador de máxima reutilización}

El método anterior resulta poco eficiente,
dado que cada muestra $S \in \mathcal{S}_i$ generada
solo contribuye a la estimación de
$\hat{\phi}_{Banzhaf}(i;v)$ y esto introduce un
factor de $n$ en la complejidad, ya que es necesario
generar un mismo número de muestras para cada dato.

En este contexto surge el concepto de estimador de máxima
reutilización (MSR) \index{MRS} \cite{dataBanzhaf}.
La idea es explotar la linealidad de la esperanza, de forma que
partiendo de \ref{ec:simpleMontecarloEstimator} llegamos a:

\begin{equation}
  \label{maximumSampleReuse}
  \phi_{Banzhaf}(i;v) = \mathbb{E}_{S \sim \text{Unif}(2^{N\setminus \{i\}})}
  [v(S \cup \{i\})] - 
  \mathbb{E}_{S \sim \text{Unif}(2^{N\setminus \{i\}})}
  [v(S)].
\end{equation}

Tomemos como ejemplo un conjunto de $m$ muestras
$\mathcal{S} = \{S_1,\dots,S_m\}$ generado de manera
uniforme. Para cada $i \in N$, podemos clasificar las
muestras de $\mathcal{S}$ en dos categorías:
\index{$\mathcal{S}$}

\begin{itemize}
  \item $\mathcal{S}_{\ni i}$: el conjunto de muestras
  que contienen el dato $i$, es decir,
  $\mathcal{S}_{\ni i} = \{S \in \mathcal{S}: i \in S\}$.
  \item $\mathcal{S}_{\not \ni  i}$: el conjunto de muestras
  que no contienen el dato $i$, esto es,
  $\mathcal{S}_{\not \ni  i} = \{S \in \mathcal{S}: i \not
  \in S\}$.
\end{itemize}

Así, para cada jugador, diferenciamos entre las muestras
que incluyen a dicho jugador y las que no. Utilizando esta
clasificación y la ecuación \ref{maximumSampleReuse},
podemos estimar $\phi_{Banzhaf}(i;v)$ de la siguiente manera: 

\begin{equation*}
  \hat{\phi}_{MSR}(i;v) = \frac{1}{|\mathcal{S}_{\ni i}|}
  \sum_{S \in \mathcal{S}_{\ni i}} v(S) -
  \frac{1}{|\mathcal{S}_{\not \ni i}|}
  \sum_{S \in \mathcal{S}_{\not \ni i}} v(S).
\end{equation*}

Este estimador se conoce como \emph{estimador de máxima
reutilización} (MSR). \index{$\hat{\phi}_{MSR}$}
\index{Estimador de máxima reutilización}

\begin{theorem}
  $\hat{\phi}_{MSR}$ es una $(\epsilon,\delta)$-aproximación
  de $\phi_{Banzhaf}$ en norma $l_p$ con  $\mathcal{O}
  (\frac{n}{\epsilon^2} \log(\frac{n}{\delta}))$ evaluaciones
  de $v$, y $\mathcal{O}(\frac{1}{\epsilon^2}
  \log{\frac{n}{\delta}})$ evaluaciones de $v$ en la norma
  $l_{\infty}$.
\end{theorem}


La demostración se puede consultar en el 
apéndice C.1.2 de \cite{dataBanzhaf}.

\

Una de las grandes ventajas que presenta el valor
de Banzhaf respecto a los demás semivalores es que
se trata del único semivalor que permite la implementación
del algoritmo MSR, como puede verse en el apéndice
C.2 de \cite{dataBanzhaf}.

El siguiente teorema nos da una estimación de la bondad del
MSR frente al resto de estimadores posibles.

\begin{theorem}
  Todo estimador aleatorio del valor de Banzhaf
  que sea una $(\epsilon,\delta)$-aproximación en norma
  $l_{\infty}$ con $\delta \in (0,\frac{1}{2})$ requiere
  al menos $\Omega(\frac{1}{\epsilon})$.
\end{theorem}

La demostración detallada se puede consultar en el apéndice
C.1.3 de \cite{dataBanzhaf}.

\

Como hemos visto anteriormente el algoritmo MSR presenta
una complejidad de $\mathcal{O}(\frac{1}{\epsilon^2}
\log(\frac{n}{\delta}))$ en la norma $l_{\infty}$.
Esto quiere decir que se aleja de la optimalidad
en un factor de $\mathcal{O}(\frac{1}{\epsilon}
\log(\frac{n}{\delta}))$.