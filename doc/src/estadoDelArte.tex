\chapter{Estado del arte}
\label{estadoDelArte}
\justifying
Desde la aparición del valor de Shapley como un método de reparto
justo de recompensas en juegos cooperativos \cite{shapleyValue},
este concepto se ha utilizado en diversos campos como la
economía \cite{libroShapley}, estudio de sistemas multiagente
\cite{fatima} e incluso en áreas en las que su aplicación
puede resultar menos evidente como la genética
\cite{genes}. Esta versatilidad se debe, en parte, a su 
sólida base matemática y a sus intuitivas interpretaciónes,
entre las que podemos resaltar:

\begin{itemize}
  \item Pago justo: El valor de Shapley de un jugador es la cantidad
  que este debería recibir si las recompensas se distribuyesen de
  manera que los jugadores fueran recompensados en función de su
  contribución a la recompensa total.

  \item Poder de negociación: El valor de Shapley puede
  interpretarse como una medida del poder de negociación de un jugador.
  Un jugador tiene más poder de negociación si su ausencia causa
  una mayor disminución en la recompensa total que se puede obtener.
\end{itemize}

El valor de Shapley se basa en una serie de axiomas fundamentales,
que garantizan propiedades como su equidad, eficiencia y simetría.
En economía relajar estos axiomas, con el objetivo de dar lugar
a nuevas formas de reparto de recomenpensa, ha sido uno de
los principales temas de estudio. Ejemplos de esto serían el
concepto de semivalor, el cual se obtiene al eliminar el
axioma de eficiencia \cite{Dubey, Dubey2}.
Este axioma asegura que la suma de los valores de
todos los jugadores sea igual a la recompensa total disponible.
Por tanto, al suprimirlo, los semivalores permiten cierta
flexibilidad en este aspecto, lo que puede ser útil en
situaciones en las que no todos los beneficios se pueden distribuir.
Del mismo modo, eliminar el axioma de simetría, que
establece que dos jugadores con igual contribución deben recibir
igual recompensa, lleva al valor de Banzhaf \cite{banzhaf},
que proporciona una medida de poder de un jugador basada
en cuánto puede cambiar el resultado de un juego al unirse o
abandonar una coalición. Cabe destacar que tanto el valor de
Shapley como el valor de Banzhaf pueden obtenerse como
particularizaciones del concepto de semivalor.

\

Debido a la naturaleza combinatoria del valor de Shapley, el cálculo
de este es altamente costoso a nivel computacional y resulta en una
tarea cuya complejidad crece exponencialmente al aumentar el
número de jugadores. Es por esto que surgen métodos de estimación del
mismo, la mayoría de estos métodos se basan en técnias de Montecarlo.
En 1960, Irwin Mann y el propio Shapley mencionan las estimaciones
basada en muestreo de permutaciones \cite{permutationSampling}. Pero
no es hasta 2015 que se lleva a cabo un análisis de la complejidad
a la hora de muestrear usando dicha técnica \cite{maleki}.
Tras esto, en 2019, Covert propone un nuevo método de estimación basado
en la técnica de muestreo por importancia que mejora los actuales
\cite{covert}. Cabe destacar el \textit{ApproShapley} propuesto en
{\cite{Tejada}}  desarrollado por los profesores J.Castro,
D.Gómez y J.Tejada de la UCM.

\

Una de las primeras apariciones del valor de Shapley en el campo
del aprendizaje computacional data del año 2005 como un método de
selección de variables \cite{featureSelection}. Más tarde, en
2017, se utiliza en el diseño del marco SHAP \cite{featureImportance},
enfocado en la evaluación de la importancia de variables en modelos de
predicción. Sin embargo, no es asta el año 2019 en el que se introduce
como una alternativa a los métodos de valoración de datos del
momento \cite{dataShapley}, acuñándose así el concepto 
\textit{Data Shapley}.

\

% Este párrafo a lo mejor lo quitamos.
Al igual que el cálculo del valor de Shapley, el cálculo de
\textit{Data Shapley} es altamente costoso a nivel computacional,
por lo que surgen también varios métodos de aproximación, entre
los que podemos destacar \textit{Group Testing} \cite{looFuck},
métodos de aproximación y cálculo exacto para problemas en los
que se aplican métodos como KNN o derivados de este \cite{knn} y
diversas técnicas basadas en métodos de Montecarlo como las vistas
en \cite{dataShapley}.

Cuando se prueba la eficacia de \textit{Data Shapley} en problemas de
aprendizaje computacional como la detección de outliers o datos corruptos
{\cite{dataShapley}}, la investigación sigue el mismo camino que años
antes en teoría de juegos y se empiezan a reciclar conceptos como los
semivalores o el valor de Banzhaf. Es en la línea de los semivalores que
en 2022 surge \textit{Beta Shapley} {\cite{betaShapley}}, una
generalización de \textit{data Shapley} que supera los
resultados de los métodos más actuales de valoración de datos
en varias tareas como son detección de muestras mal etiquetadas
y selección de puntos problemáticos a la hora de entrenar un modelo.

\

En 2023, aparece \textit{data Banzhaf} \cite{dataBanzhaf},
un nuevo método de valoración de datos derivado del valor de Banzhaf.
Este nuevo método surge como una solución a la falta de robustez
de las herramientas de valoración de datos existentes, falta de
robustez causada en parte por factores difíciles de controlar como
la aleatoriedad del método del descenso del gradiente
estocástico, el cual es ampliamente usado hoy en día.
Para solventar esta falta de robustez se apoya en el
concepto de \textit{Safety Margin}, y demuestra que el valor de
Banzhaf es el semivalor con mayor \textit{Safety Margin}. 
\textit{Data Banzhaf} supera a los existentes métodos de valoración
basados en semivalores en varias tareas de aprendizaje automático.

\

Aunque en este trabajo nos centramos en
métodos de valoración de datos que se derivan directamente de
conceptos de la teoría de juegos, existen otros métodos
que, aún utilizando teoría de juegos, siguen enfoques relativamente
distintos. Podemos destacar algunas obras como \cite{tay} en la que
sugieren un método de valoración de datos para modelos
generativos que utiliza la discrepancia media máxima (MMD) entre
la fuente de datos y la distribución real de datos. En \cite{xu}
proponen una medida de diversidad, llamada volumen robusto (RV),
para valorar las fuentes de datos. La robustez de RV se discute
en términos de la estabilidad frente a la replicación de datos.
Finalmente en \cite{amiri} utilizan diferencias estadísticas
entre los datos de origen y un conjunto de datos de referencia
como la métrica de valoración. Estas diferencias estadísticas
se miden mediante el uso de los conceptos de diversidad y
relevancia de los datos previamente comentados.

\

Como algo casi anecdótico hay literatura en la que se lleva a
cabo valoración de datos mediante aprendizaje por refuerzo
\cite{reinforcement}. En dicho trabajo, se utiliza una red
neuronal profunda para obtener un estimador de la probabiliadad
de cada dato de ser usado en el entrenamiendo del modelo
de predicción. Este estimador se obtiene mediante aprendizaje
por refuerzo.