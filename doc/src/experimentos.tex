\chapter{Análisis Experimental}

\section*{Introducción}
Esta sección detalla los experimentos efectuados para evaluar la
eficacia de los métodos de valoración de datos que
hemos investigado. Cabe destacar el uso de las herramientas
\textit{pyDVL}, una librería de
\href{https://www.appliedai.de/en/}{appliedAI} y \textit{DVC},
herramienta de control del versiones para proyectos de ciencia
de datos.

Se trabajará con conjuntos de datos estandar que se usan normalmente
en la literatura para la evaluación de métodos de clasificación
\cite{dataBanzhaf}.
Los dataset se pueden observar en la tabla \ref{tab:datasets}.

Evaluaremos los métodos de valoración de datos en dos
tareas de aprendizaje automático, la detección de puntos
mal etiquetados y el entrenamiento ponderado.
Compararemos los resultados obtenidos por LOO, data Banzhaf,
data Shapley, Beta(1, 4), Beta(1, 16),
Beta(4, 1), Beta(16, 1). % Los nombres de los Beta no están bien.
En cuanto a los conjuntos de datos, se utilizarán siete
conjuntos de datos que se usan habitualmente en la literatura
como \textit{benchmarks} para la valoración de datos 
\cite{dataBanzhaf}.



\subsection*{pyDVL}

\href{https://aai-institute.github.io/pyDVL/0.7.0/}{pyDVL} es una
herramienta en proceso de mejora, actualmente en su versión
\texttt{0.7.0}. \textit{pyDVL} agrupa algoritmos destinados al
cálculo de valores de datos, la mayoría basados en la teoría de
juegos cooperativos.

\textit{pyDVL} nos permite calcular de forma sencilla y
automática los valores de datos de un conjunto de datos
usando distintos métodos de valoración, métodos de
muestreo, estimadores... 

pyDVL divide el  proceso del cálculo de valores de datos
en tres sencillos pasos:
\begin{enumerate}
    \item Construcción del pyDVL Dataset a partir de tus datos.
    \item Se crea la utilidad. Concepto abstracto que vincula
    el modelo que se usará, el conjunto de datos y la métrica
    de error.
    \item Se calculan los valores de datos, con el método que
    se desee.
\end{enumerate}

% \begin{figure}[h!]
%     \centering
%     \begin{lstlisting}
%     import pyDVL
%     dataset = pyDVL.Dataset(data)
%     utility = pyDVL.Utility(model, dataset, metric)
%     values = utility.calculate_values(method="desired_method")
%     \end{lstlisting}
%     \captionof{figure}{Ejemplo de código usando pyDVL.}
%     \label{fig:pydvl_code}
% \end{figure}







\begin{table}[ht]
    \centering
    \resizebox{0.6\columnwidth}{!}{%
    \begin{tabular}{lc}
        \hline
        \multicolumn{1}{c}{\textbf{Dataset}} & \textbf{Source}                                                       \\ \hline
        \rowcolor[HTML]{FFFFFF} 
        Click                                & \href{https://www.openml.org/d/1218}{https://www.openml.org/d/1218}   \\
        \rowcolor[HTML]{FFFFFF} 
        Apsfail                              & \href{https://www.openml.org/d/41138}{https://www.openml.org/d/41138} \\
        Phoneme                              & \href{https://www.openml.org/d/1489}{https://www.openml.org/d/1489}   \\
        Wind                                 & \href{https://www.openml.org/d/847}{https://www.openml.org/d/847}     \\
        Pol                                  & \href{https://www.openml.org/d/722}{https://www.openml.org/d/722}     \\
        CPU                                  & \href{https://www.openml.org/d/761}{https://www.openml.org/d/761}     \\
        2DPlanes                             & \href{https://www.openml.org/d/727}{https://www.openml.org/d/727}     \\ \hline
    \end{tabular}%
    }
    \caption{Conjuntos de datos usados en los experimentos}
    \label{tab:datasets}
\end{table}


\subsection*{DVC}

\textit{DVC}, acrónimo de Data Version Control, es una
herramienta diseñada para la ciencia de datos que se
integra con herramientas de ingeniería de software ya
existentes. Su función es asistir a los equipos de machine
learning en la administración de grandes conjuntos de datos,
garantizar la reproducibilidad de los proyectos y potenciar
la colaboración. \textit{DVC} es compatible con cualquier
terminal y puede ser usado como una biblioteca
de Python.

El objetivo central de \textit{DVC} es brindar una experiencia
al estilo Git para estructurar datos, modelos y experimentos en
proyectos de Ciencia de Datos y Aprendizaje Automático.

Para más detalles, consulta la
\href{https://dvc.org/doc/user-guide}{Documentación de DVC}.

\section{Detección de ruido sintético}
Compararemos la habilidad de detección de

We investigate the ability of differ-
ent data value notions in detecting mislabeled points under
noisy utility functions. We generate noisy labeled samples
by flipping labels for a randomly chosen 10\% of training
data points. We mark a data point as a mislabeled one if its
data value is less than 10 percentile of all data value scores.
We use F1-score as the performance metric for mislabeling
detection.

\section{Entrenamiento ponderado}
% Este nombre hay que verlo
En este experimento se usan los \textit{data values}
como pesos para el entrenamiento de un modelo.



We now examine how the data value-based importance
weight can be applied to subsample data points. We
train a model with 25\% of the given dataset by using
the importance weight max for
the i-th sample. With this importance weight, data
points with higher values are more likely to be selected
and data points with negative values are not used. We
train a classifier to minimize the weighted loss then
evaluate the accuracy on the held-out test dataset.

Se hace algo similar a esto: Explicar lo que hace el fit
We weight each training point by normalizing
the associated data value between [0,1]. Then, during train-
ing, each training sample will be selected with a probability
equal to the assigned weight. As a result, data points with
a higher value are more likely to be selected in the random
mini-batch of SGD, and data points with a lower value are
rarely used. We train a neural network classifier to mini-
mize the weighted loss, and then evaluate the accuracy on
the held-out test dataset. As Table 1 shows, Data Banzhaf
outperforms other baselines.

\section{Resultados}


