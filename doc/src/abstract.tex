\chapter*{Resumen}
\justifying

Este trabajo se enfoca en la valoración de datos mediante la
teoría de juegos. El principal objetivo es comparar los
diferentes métodos presentes en la literatura actual,
prestando especial atención al citado en \cite{dataBanzhaf}.
Además, se busca realizar esta comparación implementando
buenas prácticas tanto en ciencia de datos como en ingeniería
de software, con el fin de obtener un producto final
reproducible y de alta calidad. Para llevar a cabo los
experimentos necesarios en la comparación de estos
métodos, se emplea la librería pyDVL, que implementa los
algoritmos analizados. El código desarrollado está
disponible en \href{https://github.com/Fran-AM/TFM}{Fran-AM/TFM}.

A lo largo del estudio, se realiza una revisión exhaustiva de la
literatura actual sobre el tema, culminando en el método
\textit{Data Banzhaf}, el cual ha demostrado obtener los mejores
resultados en tareas como la detección de datos mal etiquetados
y el entrenamiento ponderado, experimentos que replicaremos en
nuestro trabajo.

Los resultados obtenidos no fueron los esperados, dado que
\textit{Data Banzhaf} no logró superar a otros métodos
estudiados. Sin embargo, existe una razón para ello: no se
pudo utilizar el estimador MSR, que era el que ofrecía mejores
resultados de convergencia. La limitada capacidad de cómputo
podría ser una razón por la cual los resultados no son
completamente confiables.





% Marco teórico

El capítulo se adentra en la teoría de juegos, un campo
matemático que analiza cómo las decisiones de un individuo
o jugador pueden influir en los resultados de otros jugadores
en situaciones interactivas. Se destaca la importancia de los
juegos cooperativos, donde los jugadores tienen la capacidad
de comunicarse y establecer acuerdos.

Posteriormente, el capítulo aborda la relevancia de la
valoración de datos en la sociedad actual. Se subraya cómo
los datos se han convertido en un recurso crucial en
diversos sectores y cómo es esencial determinar su valor.
Se discuten diferentes perspectivas desde las cuales se
puede valorar un dato, como su valor inherente o el valor
basado en su uso.

El enfoque principal del capítulo es entender cómo se puede
determinar el valor de los datos y cómo este valor puede
influir en decisiones y resultados en diferentes contextos.

El tercer capítulo se centra en comprender cómo los valores
asignados a los datos pueden influir en el entrenamiento de
modelos y, en consecuencia, en las decisiones que se derivan
de estos modelos. Se detallan los experimentos realizados,
los conjuntos de datos empleados y las herramientas esenciales
que respaldaron la investigación.

Se mencionan varios conjuntos de datos que se utilizan
comúnmente en la literatura como puntos de referencia para
estudios similares. Estos conjuntos de datos contienen
información etiquetada y se han preprocesado para equilibrar
el número de muestras de cada clase.

En cuanto a las herramientas, se destaca el uso de pyDVL,
una librería que engloba diversos algoritmos enfocados al
cálculo de valores de datos, y DVC, una herramienta de control
de versiones específica para proyectos de ciencia de datos.

El capítulo también hace referencia a un repositorio Git,
"repoGit", donde se puede consultar todo el código desarrollado
para este trabajo.