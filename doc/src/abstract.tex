\chapter{Resumen}
\justifying

Este trabajo  se centra en la valoración de
datos basada en teoría de juegos. Se comienza llevando a cabo
una revisión exhaustiva de la literatura actual sobre el tema,
introduciendo posteriormente los
conceptos básicos necesarios para la comprensión de este, y
destacando la relevancia de los juegos cooperativos, los cuales
actuarán como punto de unión entre la valoración de datos y la teoría
de juegos en este estudio.

\

El objetivo principal es comparar los diferentes métodos
descritos en la literatura, haciendo especial énfasis
en \textit{Data Banzhaf} \cite{dataBanzhaf}. A lo largo
del estudio, se aborda el concepto de \textit{safety margin},
el cual servirá para medir la robustez de los métodos de valoración.
Cualidad en la que \textit{data Banzhaf} ha demostrados sobresalir
por encima del resto de semivalores.

\

Otro objetivo que perseguía este trabajo era familiarizarse
con buenas prácticas tanto en ciencia de datos como en ingeniería
de software. Para ello, se ha utilizado DVC como control de versiones
especializado para proyectos de ML y se han seguido las recomendaciones
dadas en  \href{https://paperswithcode.com/rc2022}
{ML Reproducibility Challenge}
de \href{https://paperswithcode.com/}{papers with code}.
Con esto se ha conseguido un código reproducible, una fundamental
en la ciencia, pero no muy común en ciencia de datos.
El código  se encuentra disponible en
\href{https://github.com/Fran-AM/TFM}{Fran-AM/TFM}.

\

Finalmente, los métodos estudiados se comparan en dos
tareas de ML que suelen usarse como métrica de rendimiento
para este tipo de estudios: detección de datos mal etiquetados y
entrenamiento ponderado de modelos. Para estos experimentos,
se seleccionan varios conjuntos de datos comúnmente
referenciados en la literatura en trabajos similares.

